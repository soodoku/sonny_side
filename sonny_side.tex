\documentclass[]{article}
\usepackage{lmodern}
\usepackage{amssymb,amsmath}
\usepackage{ifxetex,ifluatex}
\usepackage{fixltx2e} % provides \textsubscript
\ifnum 0\ifxetex 1\fi\ifluatex 1\fi=0 % if pdftex
  \usepackage[T1]{fontenc}
  \usepackage[utf8]{inputenc}
\else % if luatex or xelatex
  \ifxetex
    \usepackage{mathspec}
  \else
    \usepackage{fontspec}
  \fi
  \defaultfontfeatures{Ligatures=TeX,Scale=MatchLowercase}
\fi
% use upquote if available, for straight quotes in verbatim environments
\IfFileExists{upquote.sty}{\usepackage{upquote}}{}
% use microtype if available
\IfFileExists{microtype.sty}{%
\usepackage{microtype}
\UseMicrotypeSet[protrusion]{basicmath} % disable protrusion for tt fonts
}{}
\usepackage[margin=1in]{geometry}
\usepackage{hyperref}
\hypersetup{unicode=true,
            pdftitle={Son Bias in the US: Evidence from Business Names},
            pdfauthor={Guillioli, Walter and Sood, Gaurav},
            pdfborder={0 0 0},
            breaklinks=true}
\urlstyle{same}  % don't use monospace font for urls
\usepackage{color}
\usepackage{fancyvrb}
\newcommand{\VerbBar}{|}
\newcommand{\VERB}{\Verb[commandchars=\\\{\}]}
\DefineVerbatimEnvironment{Highlighting}{Verbatim}{commandchars=\\\{\}}
% Add ',fontsize=\small' for more characters per line
\usepackage{framed}
\definecolor{shadecolor}{RGB}{248,248,248}
\newenvironment{Shaded}{\begin{snugshade}}{\end{snugshade}}
\newcommand{\AlertTok}[1]{\textcolor[rgb]{0.94,0.16,0.16}{#1}}
\newcommand{\AnnotationTok}[1]{\textcolor[rgb]{0.56,0.35,0.01}{\textbf{\textit{#1}}}}
\newcommand{\AttributeTok}[1]{\textcolor[rgb]{0.77,0.63,0.00}{#1}}
\newcommand{\BaseNTok}[1]{\textcolor[rgb]{0.00,0.00,0.81}{#1}}
\newcommand{\BuiltInTok}[1]{#1}
\newcommand{\CharTok}[1]{\textcolor[rgb]{0.31,0.60,0.02}{#1}}
\newcommand{\CommentTok}[1]{\textcolor[rgb]{0.56,0.35,0.01}{\textit{#1}}}
\newcommand{\CommentVarTok}[1]{\textcolor[rgb]{0.56,0.35,0.01}{\textbf{\textit{#1}}}}
\newcommand{\ConstantTok}[1]{\textcolor[rgb]{0.00,0.00,0.00}{#1}}
\newcommand{\ControlFlowTok}[1]{\textcolor[rgb]{0.13,0.29,0.53}{\textbf{#1}}}
\newcommand{\DataTypeTok}[1]{\textcolor[rgb]{0.13,0.29,0.53}{#1}}
\newcommand{\DecValTok}[1]{\textcolor[rgb]{0.00,0.00,0.81}{#1}}
\newcommand{\DocumentationTok}[1]{\textcolor[rgb]{0.56,0.35,0.01}{\textbf{\textit{#1}}}}
\newcommand{\ErrorTok}[1]{\textcolor[rgb]{0.64,0.00,0.00}{\textbf{#1}}}
\newcommand{\ExtensionTok}[1]{#1}
\newcommand{\FloatTok}[1]{\textcolor[rgb]{0.00,0.00,0.81}{#1}}
\newcommand{\FunctionTok}[1]{\textcolor[rgb]{0.00,0.00,0.00}{#1}}
\newcommand{\ImportTok}[1]{#1}
\newcommand{\InformationTok}[1]{\textcolor[rgb]{0.56,0.35,0.01}{\textbf{\textit{#1}}}}
\newcommand{\KeywordTok}[1]{\textcolor[rgb]{0.13,0.29,0.53}{\textbf{#1}}}
\newcommand{\NormalTok}[1]{#1}
\newcommand{\OperatorTok}[1]{\textcolor[rgb]{0.81,0.36,0.00}{\textbf{#1}}}
\newcommand{\OtherTok}[1]{\textcolor[rgb]{0.56,0.35,0.01}{#1}}
\newcommand{\PreprocessorTok}[1]{\textcolor[rgb]{0.56,0.35,0.01}{\textit{#1}}}
\newcommand{\RegionMarkerTok}[1]{#1}
\newcommand{\SpecialCharTok}[1]{\textcolor[rgb]{0.00,0.00,0.00}{#1}}
\newcommand{\SpecialStringTok}[1]{\textcolor[rgb]{0.31,0.60,0.02}{#1}}
\newcommand{\StringTok}[1]{\textcolor[rgb]{0.31,0.60,0.02}{#1}}
\newcommand{\VariableTok}[1]{\textcolor[rgb]{0.00,0.00,0.00}{#1}}
\newcommand{\VerbatimStringTok}[1]{\textcolor[rgb]{0.31,0.60,0.02}{#1}}
\newcommand{\WarningTok}[1]{\textcolor[rgb]{0.56,0.35,0.01}{\textbf{\textit{#1}}}}
\usepackage{graphicx,grffile}
\makeatletter
\def\maxwidth{\ifdim\Gin@nat@width>\linewidth\linewidth\else\Gin@nat@width\fi}
\def\maxheight{\ifdim\Gin@nat@height>\textheight\textheight\else\Gin@nat@height\fi}
\makeatother
% Scale images if necessary, so that they will not overflow the page
% margins by default, and it is still possible to overwrite the defaults
% using explicit options in \includegraphics[width, height, ...]{}
\setkeys{Gin}{width=\maxwidth,height=\maxheight,keepaspectratio}
\IfFileExists{parskip.sty}{%
\usepackage{parskip}
}{% else
\setlength{\parindent}{0pt}
\setlength{\parskip}{6pt plus 2pt minus 1pt}
}
\setlength{\emergencystretch}{3em}  % prevent overfull lines
\providecommand{\tightlist}{%
  \setlength{\itemsep}{0pt}\setlength{\parskip}{0pt}}
\setcounter{secnumdepth}{0}
% Redefines (sub)paragraphs to behave more like sections
\ifx\paragraph\undefined\else
\let\oldparagraph\paragraph
\renewcommand{\paragraph}[1]{\oldparagraph{#1}\mbox{}}
\fi
\ifx\subparagraph\undefined\else
\let\oldsubparagraph\subparagraph
\renewcommand{\subparagraph}[1]{\oldsubparagraph{#1}\mbox{}}
\fi

%%% Use protect on footnotes to avoid problems with footnotes in titles
\let\rmarkdownfootnote\footnote%
\def\footnote{\protect\rmarkdownfootnote}

%%% Change title format to be more compact
\usepackage{titling}

% Create subtitle command for use in maketitle
\providecommand{\subtitle}[1]{
  \posttitle{
    \begin{center}\large#1\end{center}
    }
}

\setlength{\droptitle}{-2em}

  \title{Son Bias in the US: Evidence from Business Names}
    \pretitle{\vspace{\droptitle}\centering\huge}
  \posttitle{\par}
    \author{Guillioli, Walter and Sood, Gaurav}
    \preauthor{\centering\large\emph}
  \postauthor{\par}
      \predate{\centering\large\emph}
  \postdate{\par}
    \date{1/15/2020}


\begin{document}
\maketitle

\hypertarget{abstract}{%
\subsection{Abstract}\label{abstract}}

Do we want one? Maybe write at the end.

\hypertarget{introduction}{%
\subsection{1. Introduction}\label{introduction}}

NEEDS WORK. Will write at the end. We estimate preference for passing on
businesses to sons by examining how common words son and sons are
compared to daughter and daughters in the names of businesses. intro. =
bias against women is common one way bias manifests itself is that
people are less likely to transfer their businesses to their daughters
we exploit a novel publicly available administrative source of data

\hypertarget{data-and-methods-overview}{%
\subsection{2. Data and Methods
Overview}\label{data-and-methods-overview}}

Our data acquisision process is divided in two steps which are briefly
explained below.

\hypertarget{acquire-business-names-across-us-states}{%
\subsubsection{2.1 Acquire Business Names across US
states}\label{acquire-business-names-across-us-states}}

In the United States, businesses have to register with their state and
all states provide a website to search for business names. The
functionality of these websites vary by state which made the data
acquisition harder. We began by searching for businesses with the words
son(s) and daughter(s) on their names. These wasn't a trivial process
for several reasons and some are worth highlighting here along with the
solution.

\hypertarget{a-search-results-for-sons-are-inflated}{%
\paragraph{a) Search results for son(s) are
inflated}\label{a-search-results-for-sons-are-inflated}}

Search results for son(s) are inflated mainly for three reasons. (a) son
is part of many English words, from names such as Jason and Robinson to
ordinary English words like mason (which can also be a name), (b) son is
a Korean name and (c) some businesses use the word son playfully; for
instance, son is a homonym of sun and some people use that to create
names like son of a beach. We address (a) by cleaning the data using
regular expressions to only look for exact matches of son and sons.

\hypertarget{b-limits-in-the-number-of-results-shown}{%
\paragraph{b) Limits in the number of results
shown}\label{b-limits-in-the-number-of-results-shown}}

Some states limit the number of search results. For example, Alabama
only displays up to 1,000 results. This is tricky because we know there
are at least 1,000 companies with son(s) on their name but we don't know
how many. In this case we can only derive a conservative estimate for
the ratio of companies with son vs daughter and we note that on the
results. In order to increase the number of samples in this case we do
two searches, one for son and one for sons. we then combine the results
knowing there might be some overlap but we dedup these before the
analysis.

\hypertarget{c-technological-challenges-in-data-acquisition}{%
\paragraph{c) Technological challenges in data
acquisition}\label{c-technological-challenges-in-data-acquisition}}

The technologies used for these websites is different. In some cases we
are able to simply copy and paste the results to our computer for
analysis. But in other cases more sophisticated scrapping tools were
built to parse and download the data using packages like rvest in R and
selenium in Python.

After acquirign the number of companies with the word son(s) and
daughter(s) on their names we calcualted the son/daughter ratio which is
the estimate of most concern in this paper.

\hypertarget{additional-state-information}{%
\subsubsection{2.2 Additional State
Information}\label{additional-state-information}}

Additionaly, we enriched our dataset by acquiring state data from other
sources to profile the results. These new attributes include: US Region
of the state, US Division of the state, population of the state, GPD of
the state, political party of the state and number of establishments on
each state. The sources are identified in the References section of this
paper.

\hypertarget{final-dataset-used-in-analysis}{%
\subsubsection{2.3 Final Dataset used in
Analysis}\label{final-dataset-used-in-analysis}}

Due to the challenges outlined above some care is needed when
interpreting the results. All in all, we were able to acquire data for
36 states. Though not all 50 states were covered we believe we have a
good representation of the United States since these 36 states represent
69.9\% of the US population, 71.2\% of the US GDP and 71.\% of the
registered establishments.

The data and scripts used are posted here:
\url{https://github.com/soodoku/sonny_side}

A sample of the final dataset used for analysis is displayed here:

NOTE: This table looks horrible. See if I can do stargazer package here?

\begin{verbatim}
##          Name estimate  son daughter    Region           Division
## 1     Alabama        7  884      126     South East South Central
## 2      Alaska       11  246       22      West            Pacific
## 4    Arkansas       17 1482       87     South West South Central
## 5  California       24 3609      150      West            Pacific
## 7 Connecticut       20  875       43 Northeast        New England
## 9     Florida        4  729      176     South     South Atlantic
##   popestimate2019pct presidentialelection2016 gdp_pct establishments_pct
## 1        0.014937826               Republican   0.011        0.012836617
## 2        0.002228693               Republican   0.003        0.002716876
## 4        0.009193908               Republican   0.006        0.008457416
## 5        0.120376189               Democratic   0.145        0.118909506
## 7        0.010861846               Democratic   0.013        0.011525938
## 9        0.065433123               Republican   0.051        0.070408815
\end{verbatim}

Question for Gaurav -- do we want to do some univariate EDA and
description? I doubt it.

\hypertarget{results}{%
\subsection{3. Results}\label{results}}

In all, we find that a conservative estimate of son to daughter ratio is
between 2 to 1 to 72 to 1 across the 36 states where we have data with a
median of 12 to 1. This is displayed in the figure below.

\includegraphics{sonny_side_files/figure-latex/unnamed-chunk-3-1.pdf}

NOTE: If the above feels too busy we could show the top 10 only like
this:

\includegraphics{sonny_side_files/figure-latex/unnamed-chunk-4-1.pdf}

We know proceed to explore how these results vary by location of the
state in the United States, by the state population and political party
and by its GDP and number of bueinsss establishments.

NOTE: I CAN BRING \% OF MAILES VS FEMALES IN STATE TO SEE IF ANYTHING
THERE? OR OTHER VARIABLES TO CORRELATE. MAYBE LATER.

\hypertarget{differences-by-us-region}{%
\subsubsection{3.1 Differences by US
Region}\label{differences-by-us-region}}

When we look at the estimate of son to daughter ratio by Region in USA
we see states in the Midwest and Northeast with higher ratios when
comparing to the West and particularly the South.The biggest gap is
Midwest vs South with a median ratio of 24.0 vs 5.5 respectively.

\includegraphics{sonny_side_files/figure-latex/unnamed-chunk-6-1.pdf}

\hypertarget{relationship-with-the-population-size-of-the-state}{%
\subsubsection{3.2 Relationship with the Population size of the
State}\label{relationship-with-the-population-size-of-the-state}}

When we look at the estimate ratio of son vs daughter with the
population fo the state we don't see any relationship between these data
points. In fact the correlation if basically zero as seen belown.

\includegraphics{sonny_side_files/figure-latex/unnamed-chunk-8-1.pdf}

\hypertarget{relationship-with-the-gdp-of-the-state}{%
\subsubsection{3.3 Relationship with the GDP of the
State}\label{relationship-with-the-gdp-of-the-state}}

We also looked at how the differences of the ratio of business names
using son vs daughter could vary by state as it relates to the size of
the state in terms of percentage of the gross domesic product (GDP) of
the country. We didn't find any evidence of relationship between these
two with a correlation of basically zero, as seen in this figure.

\includegraphics{sonny_side_files/figure-latex/unnamed-chunk-10-1.pdf}

\hypertarget{relationship-with-the-number-of-establishments-of-the-state}{%
\subsubsection{3.4 Relationship with the number of Establishments of the
State}\label{relationship-with-the-number-of-establishments-of-the-state}}

We also obtained data from the census organization from US that offers
the number of registered establishments in the USA, this is not
necesarily the same as the number of business companies registered per
state but since obtaining that exact number wasn't not possible we use
this as a proxy. Again, no evidence or correlation is seen per figure
below.

\includegraphics{sonny_side_files/figure-latex/unnamed-chunk-12-1.pdf}

\hypertarget{relationship-with-the-major-political-party-of-the-state}{%
\subsubsection{3.5 Relationship with the major Political Party of the
State}\label{relationship-with-the-major-political-party-of-the-state}}

Finally we looked at the voting data from the 2016 elections and compare
how the ratio of son vs daughter is when separating states with a
majority of Democratics vs Republicans. As can be seen below the ratio
tends to be higher on Democratics states with a median of 17 vs a median
of 11 for Republican states.

\includegraphics{sonny_side_files/figure-latex/unnamed-chunk-14-1.pdf}

\hypertarget{conclusion}{%
\subsection{4. Conclusion}\label{conclusion}}

There is clearly an inclination to name businesses including the word
son(s) vs daughter(s). We found evidence for 36 states that that a
conservative estimate of son to daughter ratio is between 2 to 1 to 72
to 1 across the 36 states where we have data with a median of 12 to 1.
Despite not having data for 50 states we feel this is a good
representation of the whole country since these 36 states represent
69.9\% of the US population, 71.2\% of the US GDP and 71.\% of the
registered establishments.

Although we didn't find any relatioship with the size of the states in
terms of GDP, population or number of establishments we do see some
differences across regions and political parties dominating the state.
We cannot conclude any causality because of this but further exploration
is recommended.

\hypertarget{references}{%
\subsection{References}\label{references}}

\begin{enumerate}
\def\labelenumi{\arabic{enumi}.}
\item
  Kingl, Arvid. Web Scraping in R: rvest Tutorial.
  \url{https://www.datacamp.com/community/tutorials/r-web-scraping-rvest}
\item
  Halpert, Chris. US census bureau regions and divisions.
  \url{https://github.com/cphalpert/census-regions/}
\item
  United States Census Bureau. State Population Totals and Components of
  Change: 2010-2019.
  \url{https://www.census.gov/data/tables/time-series/demo/popest/2010s-state-total.html}
\item
  Wikipedia contributors. (2020, January 13). Political party strength
  in U.S. states. In Wikipedia, The Free Encyclopedia. Retrieved 03:11,
  January 20, 2020, from
  \url{https://en.wikipedia.org/w/index.php?title=Political_party_strength_in_U.S._states\&oldid=935536430}
\item
  United States Census Bureau. SUSB Historical Data.
  \url{https://www.census.gov/data/tables/time-series/econ/susb/susb-historical.html}
\item
  Kaushik, Saurav. Beginner's Guide on Web Scraping in R (using rvest)
  with hands-on example.
  \url{https://www.analyticsvidhya.com/blog/2017/03/beginners-guide-on-web-scraping-in-r-using-rvest-with-hands-on-knowledge/}
\end{enumerate}

\hypertarget{appendix}{%
\subsection{Appendix}\label{appendix}}

\hypertarget{appendix-a---states-with-number-of-companies-found-with-word-son-and-daughter-and-ratio}{%
\subsubsection{Appendix A - States with number of companies found with
word son and daughter and
ratio}\label{appendix-a---states-with-number-of-companies-found-with-word-son-and-daughter-and-ratio}}

NOTE: add column indicating which is super conservative based on limit
for search and why.

NOTE: table looks bad, stargazer package maybe?

\begin{Shaded}
\begin{Highlighting}[]
\NormalTok{d[,}\KeywordTok{c}\NormalTok{(}\StringTok{"Name"}\NormalTok{,}\StringTok{"estimate"}\NormalTok{,}\StringTok{"son"}\NormalTok{,}\StringTok{"daughter"}\NormalTok{)]}
\end{Highlighting}
\end{Shaded}

\begin{verbatim}
##              Name estimate    son daughter
## 1         Alabama        7  884.0      126
## 2          Alaska       11  246.0       22
## 4        Arkansas       17 1482.0       87
## 5      California       24 3609.0      150
## 7     Connecticut       20  875.0       43
## 9         Florida        4  729.0      176
## 10        Georgia       12 6002.0      497
## 11         Hawaii       17 1454.0       88
## 12          Idaho        2   60.0       39
## 13       Illinois       48 2324.0       48
## 14        Indiana       25 4928.0      195
## 16         Kansas        5   75.0       14
## 17       Kentucky        4   66.0       16
## 20       Maryland        2  128.0       82
## 21  Massachusetts       41 5979.0      147
## 22       Michigan       24 2265.0       93
## 23      Minnesota        2  392.0      213
## 24    Mississippi       12 1918.0      165
## 26        Montana        4  240.0       66
## 28         Nevada       72 1440.0       20
## 29  New Hampshire       27 3203.0      119
## 30     New Jersey        2  173.0       73
## 32       New York        2 1190.0      745
## 34   North Dakota       26  605.0       23
## 35           Ohio       26 2550.0      100
## 37         Oregon        4 1000.0      227
## 39   Rhode Island       17  206.0       12
## 40 South Carolina       69 4083.3       59
## 41   South Dakota       12  129.0       11
## 42      Tennessee        2  203.0      132
## 44           Utah        5   81.0       16
## 45        Vermont       22 1361.0       63
## 47     Washington       15 2424.0      161
## 48  West Virginia        2  128.0       72
## 49      Wisconsin       20  845.0       43
## 50        Wyoming       11  238.0       21
\end{verbatim}


\end{document}
