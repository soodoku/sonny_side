\documentclass[]{article}
\usepackage{lmodern}
\usepackage{amssymb,amsmath}
\usepackage{ifxetex,ifluatex}
\usepackage{fixltx2e} % provides \textsubscript
\ifnum 0\ifxetex 1\fi\ifluatex 1\fi=0 % if pdftex
  \usepackage[T1]{fontenc}
  \usepackage[utf8]{inputenc}
\else % if luatex or xelatex
  \ifxetex
    \usepackage{mathspec}
  \else
    \usepackage{fontspec}
  \fi
  \defaultfontfeatures{Ligatures=TeX,Scale=MatchLowercase}
\fi
% use upquote if available, for straight quotes in verbatim environments
\IfFileExists{upquote.sty}{\usepackage{upquote}}{}
% use microtype if available
\IfFileExists{microtype.sty}{%
\usepackage{microtype}
\UseMicrotypeSet[protrusion]{basicmath} % disable protrusion for tt fonts
}{}
\usepackage[margin=1in]{geometry}
\usepackage{hyperref}
\hypersetup{unicode=true,
            pdftitle={Son Bias in the US: Evidence from Business Names},
            pdfauthor={Walter Guillioli and Gaurav Sood},
            pdfborder={0 0 0},
            breaklinks=true}
\urlstyle{same}  % don't use monospace font for urls
\usepackage{longtable,booktabs}
\usepackage{graphicx,grffile}
\makeatletter
\def\maxwidth{\ifdim\Gin@nat@width>\linewidth\linewidth\else\Gin@nat@width\fi}
\def\maxheight{\ifdim\Gin@nat@height>\textheight\textheight\else\Gin@nat@height\fi}
\makeatother
% Scale images if necessary, so that they will not overflow the page
% margins by default, and it is still possible to overwrite the defaults
% using explicit options in \includegraphics[width, height, ...]{}
\setkeys{Gin}{width=\maxwidth,height=\maxheight,keepaspectratio}
\IfFileExists{parskip.sty}{%
\usepackage{parskip}
}{% else
\setlength{\parindent}{0pt}
\setlength{\parskip}{6pt plus 2pt minus 1pt}
}
\setlength{\emergencystretch}{3em}  % prevent overfull lines
\providecommand{\tightlist}{%
  \setlength{\itemsep}{0pt}\setlength{\parskip}{0pt}}
\setcounter{secnumdepth}{0}
% Redefines (sub)paragraphs to behave more like sections
\ifx\paragraph\undefined\else
\let\oldparagraph\paragraph
\renewcommand{\paragraph}[1]{\oldparagraph{#1}\mbox{}}
\fi
\ifx\subparagraph\undefined\else
\let\oldsubparagraph\subparagraph
\renewcommand{\subparagraph}[1]{\oldsubparagraph{#1}\mbox{}}
\fi

%%% Use protect on footnotes to avoid problems with footnotes in titles
\let\rmarkdownfootnote\footnote%
\def\footnote{\protect\rmarkdownfootnote}

%%% Change title format to be more compact
\usepackage{titling}

% Create subtitle command for use in maketitle
\providecommand{\subtitle}[1]{
  \posttitle{
    \begin{center}\large#1\end{center}
    }
}

\setlength{\droptitle}{-2em}

  \title{Son Bias in the US: Evidence from Business Names}
    \pretitle{\vspace{\droptitle}\centering\huge}
  \posttitle{\par}
    \author{Walter Guillioli and Gaurav Sood}
    \preauthor{\centering\large\emph}
  \postauthor{\par}
      \predate{\centering\large\emph}
  \postdate{\par}
    \date{22 January, 2020}


\begin{document}
\maketitle

\hypertarget{introduction}{%
\subsection{1. Introduction}\label{introduction}}

We explore bias and preference for business owners in the United States
to inherit their businesses to their sons instead of their daughters. We
do this by examining how common words ``son(s)'' compare to
``daughter(s)'' in the names of businesses.

To do this, we collect the information that states in the United States
have avaialble online as it relates to businesses already registered in
the state. This process is not trivial since all states have different
ways and rules to allowing to search for this data and in some cases
obtaining all the data is impossible.

Once we collect the companies using the words son(s) and daughter(s) we
estimate a ratio of these two numbers to measure the son to daughter
bias per state. Finally, these data is correlated with other metrics in
the state like GDP, population, geograhic location and political
majority.

\hypertarget{data-and-methods-overview}{%
\subsection{2. Data and Methods
Overview}\label{data-and-methods-overview}}

Our data acquisision process is divided in two steps which are briefly
explained below.

\hypertarget{acquire-business-names-across-us-states}{%
\subsubsection{2.1 Acquire Business Names across US
states}\label{acquire-business-names-across-us-states}}

In the United States, businesses have to register with their state and
all states provide a website to search for business names. The
functionality of these websites vary by state which made the data
acquisition harder. We began by searching for businesses with the words
son(s) and daughter(s) on their names. This is not a trivial process for
several reasons. The main challenges and nuances are:

\begin{enumerate}
\def\labelenumi{\alph{enumi})}
\tightlist
\item
  Search results for son(s) are inflated. This is mainly for three
  reasons. First, son is part of many English words, from names such as
  Jason and Robinson to ordinary English words like mason (which can
  also be a name). Second, son is a Korean name. Third, some businesses
  use the word son playfully; for instance, son is a homonym of sun and
  some people use that to create names like ``son of a beach''.
\end{enumerate}

We address the first issue by cleaning the data using regular
expressions to only look for exact matches of son and sons. We do not
deal with the other two issues but we believe the impact is minimal.

\begin{enumerate}
\def\labelenumi{\alph{enumi})}
\setcounter{enumi}{1}
\tightlist
\item
  Limits in the number of results shown. Some states show all the
  results when doing a search but some states limit the number of search
  results shown. For example, Alabama only displays up to 1,000 results.
  This has significant impact since we only know if a particular search
  for son(s) has more than 1,000 results but we don't know how many - it
  could be 10,000 or 500,000.
\end{enumerate}

To deal with this challenge, we only derive a conservative estimate for
the ratio of companies with son vs daughter and we note that on the
results. In addition, in order to increase the number of samples in some
cases we do multiple searches for son and sons and for business names
starting and containg this text. We then combine the results knowing
there might be some overlap but we dedup before the analysis.

\begin{enumerate}
\def\labelenumi{\alph{enumi})}
\setcounter{enumi}{2}
\tightlist
\item
  Technological challenges in data acquisition. In some states we are
  able to simply copy and paste the results in tabular format to our
  computer for analysis. But in other cases more sophisticated scrapping
  tools were built to parse and download the data using packages like
  rvest in R and selenium in Python.
\end{enumerate}

After acquirign the number of companies with the word son(s) and
daughter(s) on their names we calcualted the son/daughter ratio which is
the estimate of most concern in this paper.

\hypertarget{additional-state-information}{%
\subsubsection{2.2 Additional State
Information}\label{additional-state-information}}

We enriched our dataset by acquiring state data from other sources to
profile the results. These new attributes include: US Region of the
state, US Division of the state, population of the state, GPD of the
state, major political party of the state and number of establishments
on each state. The sources are identified in the References section of
this paper.

\hypertarget{final-dataset-used-in-analysis}{%
\subsubsection{2.3 Final Dataset used in
Analysis}\label{final-dataset-used-in-analysis}}

Due to the challenges outlined above some care is needed when
interpreting the results. In all, we were able to acquire data for 36
states. Though not all 50 states were covered we believe we have a good
representation of the United States since these 36 states represent
69.9\% of the US population, 71.2\% of the US GDP and 71.\% of the
registered establishments.

The data and scripts used are posted here:
\url{https://github.com/soodoku/sonny_side}

A sample of the final dataset used for analysis is displayed here:

\begin{longtable}[]{@{}lrlrrrl@{}}
\toprule
State & Son/Daughter Ratio & US Region & Population \% & GDP \% &
Establishments \% & Major Political Party\tabularnewline
\midrule
\endhead
Alabama & 7 & South & 0.015 & 0.011 & 0.013 & Republican\tabularnewline
Alaska & 11 & West & 0.002 & 0.003 & 0.003 & Republican\tabularnewline
Arkansas & 17 & South & 0.009 & 0.006 & 0.008 &
Republican\tabularnewline
California & 24 & West & 0.120 & 0.145 & 0.119 &
Democratic\tabularnewline
Connecticut & 20 & Northeast & 0.011 & 0.013 & 0.012 &
Democratic\tabularnewline
Florida & 4 & South & 0.065 & 0.051 & 0.070 & Republican\tabularnewline
\bottomrule
\end{longtable}

********** Question for Gaurav -- do we want to do some univariate EDA
and description? I doubt it.

\hypertarget{results}{%
\subsection{3. Results}\label{results}}

In all, we find that a conservative estimate of son to daughter ratio is
between 2 to 1 to 72 to 1 across the 36 states where we have data with a
median of 12 to 1. This is displayed in the figure below.

\includegraphics{sonny_side_files/figure-latex/unnamed-chunk-3-1.pdf}

We know proceed to explore how these results vary by location of the
state in the United States, by the state population and political party
and by its GDP and number of bueinsss establishments.

NOTE: I CAN BRING \% OF MAILES VS FEMALES IN STATE TO SEE IF ANYTHING
THERE? OR OTHER VARIABLES TO CORRELATE. MAYBE LATER.

\hypertarget{differences-by-us-region}{%
\subsubsection{3.1 Differences by US
Region}\label{differences-by-us-region}}

When we look at the estimate of son to daughter ratio by Region in USA
we see states in the Midwest and Northeast with higher ratios when
comparing to the West and particularly the South.The biggest gap is
Midwest vs South with a median ratio of 24.0 vs 5.5 respectively.

\includegraphics{sonny_side_files/figure-latex/unnamed-chunk-5-1.pdf}

\hypertarget{relationship-with-the-population-size-of-the-state}{%
\subsubsection{3.2 Relationship with the Population size of the
State}\label{relationship-with-the-population-size-of-the-state}}

When we look at the estimate ratio of son vs daughter with the
population fo the state we don't see any relationship between these data
points. In fact the correlation if basically zero as seen belown.

\includegraphics{sonny_side_files/figure-latex/unnamed-chunk-7-1.pdf}

\hypertarget{relationship-with-the-gdp-of-the-state}{%
\subsubsection{3.3 Relationship with the GDP of the
State}\label{relationship-with-the-gdp-of-the-state}}

We also looked at how the differences of the ratio of business names
using son vs daughter could vary by state as it relates to the size of
the state in terms of percentage of the gross domesic product (GDP) of
the country. We didn't find any evidence of relationship between these
two with a correlation of basically zero, as seen in this figure.

\includegraphics{sonny_side_files/figure-latex/unnamed-chunk-9-1.pdf}

\hypertarget{relationship-with-the-number-of-establishments-of-the-state}{%
\subsubsection{3.4 Relationship with the number of Establishments of the
State}\label{relationship-with-the-number-of-establishments-of-the-state}}

We also obtained data from the census organization from US that offers
the number of registered establishments in the USA, this is not
necesarily the same as the number of business companies registered per
state but since obtaining that exact number wasn't not possible we use
this as a proxy. Again, no evidence or correlation is seen per figure
below.

\includegraphics{sonny_side_files/figure-latex/unnamed-chunk-11-1.pdf}

\hypertarget{relationship-with-the-major-political-party-of-the-state}{%
\subsubsection{3.5 Relationship with the major Political Party of the
State}\label{relationship-with-the-major-political-party-of-the-state}}

Finally we looked at the voting data from the 2016 elections and compare
how the ratio of son vs daughter is when separating states with a
majority of Democratics vs Republicans. As can be seen below the ratio
tends to be higher on Democratics states with a median of 17 vs a median
of 11 for Republican states.

\includegraphics{sonny_side_files/figure-latex/unnamed-chunk-13-1.pdf}

\hypertarget{conclusion}{%
\subsection{4. Conclusion}\label{conclusion}}

There is clearly an inclination to name businesses including the word
son(s) vs daughter(s). We found evidence for 36 states that that a
conservative estimate of son to daughter ratio is between 2 to 1 to 72
to 1 across the 36 states where we have data with a median of 12 to 1.
Despite not having data for 50 states we feel this is a good
representation of the whole country since these 36 states represent
69.9\% of the US population, 71.2\% of the US GDP and 71.\% of the
registered establishments.

Although we didn't find any relatioship with the size of the states in
terms of GDP, population or number of establishments we do see some
differences across regions and political parties dominating the state.
We cannot conclude any causality because of this but further exploration
is recommended.

\hypertarget{references}{%
\subsection{References}\label{references}}

\begin{enumerate}
\def\labelenumi{\arabic{enumi}.}
\item
  Halpert, Chris. US census bureau regions and divisions.
  \url{https://github.com/cphalpert/census-regions/}
\item
  Kaushik, Saurav. Beginner's Guide on Web Scraping in R (using rvest)
  with hands-on example.
  \url{https://www.analyticsvidhya.com/blog/2017/03/beginners-guide-on-web-scraping-in-r-using-rvest-with-hands-on-knowledge/}
\item
  Kingl, Arvid. Web Scraping in R: rvest Tutorial.
  \url{https://www.datacamp.com/community/tutorials/r-web-scraping-rvest}
\item
  United States Census Bureau. State Population Totals and Components of
  Change: 2010-2019.
  \url{https://www.census.gov/data/tables/time-series/demo/popest/2010s-state-total.html}
\item
  United States Census Bureau. SUSB Historical Data.
  \url{https://www.census.gov/data/tables/time-series/econ/susb/susb-historical.html}
\item
  Wikipedia contributors. (2020, January 13). Political party strength
  in U.S. states. In Wikipedia, The Free Encyclopedia. Retrieved 03:11,
  January 20, 2020, from
  \url{https://en.wikipedia.org/w/index.php?title=Political_party_strength_in_U.S._states\&oldid=935536430}
\end{enumerate}

\hypertarget{appendix}{%
\subsection{Appendix}\label{appendix}}

\hypertarget{a---states-with-number-of-companies-found-with-word-son-and-daughter-and-ratio}{%
\subsubsection{A - States with number of companies found with word son
and daughter and
ratio}\label{a---states-with-number-of-companies-found-with-word-son-and-daughter-and-ratio}}

\begin{longtable}[]{@{}lrrrl@{}}
\toprule
State & Son/Daughter Ratio & \# Companies with son & \# Companies with
daughter & Conservative Estimate\tabularnewline
\midrule
\endhead
Alabama & 7 & 884 & 126 & Yes\tabularnewline
Alaska & 11 & 246 & 22 &\tabularnewline
Arkansas & 17 & 1482 & 87 &\tabularnewline
California & 24 & 3609 & 150 & Yes\tabularnewline
Connecticut & 20 & 875 & 43 &\tabularnewline
Florida & 4 & 729 & 176 & Yes\tabularnewline
Georgia & 12 & 6002 & 497 &\tabularnewline
Hawaii & 17 & 1454 & 88 & Yes\tabularnewline
Idaho & 2 & 60 & 39 &\tabularnewline
Illinois & 48 & 2324 & 48 &\tabularnewline
Indiana & 25 & 4928 & 195 &\tabularnewline
Kansas & 5 & 75 & 14 & Yes\tabularnewline
Kentucky & 4 & 66 & 16 & Yes\tabularnewline
Maryland & 2 & 128 & 82 & Yes\tabularnewline
Massachusetts & 41 & 5979 & 147 &\tabularnewline
Michigan & 24 & 2265 & 93 &\tabularnewline
Minnesota & 2 & 392 & 213 & Yes\tabularnewline
Mississippi & 12 & 1918 & 165 &\tabularnewline
Montana & 4 & 240 & 66 & Yes\tabularnewline
Nevada & 72 & 1440 & 20 &\tabularnewline
New Hampshire & 27 & 3203 & 119 &\tabularnewline
New Jersey & 2 & 173 & 73 & Yes\tabularnewline
New York & 2 & 1190 & 745 & Yes\tabularnewline
North Dakota & 26 & 605 & 23 & Yes\tabularnewline
Ohio & 26 & 2550 & 100 &\tabularnewline
Oregon & 4 & 1000 & 227 & Yes\tabularnewline
Rhode Island & 17 & 206 & 12 &\tabularnewline
South Carolina & 69 & 4083 & 59 & Yes\tabularnewline
South Dakota & 12 & 129 & 11 &\tabularnewline
Tennessee & 2 & 203 & 132 & Yes\tabularnewline
Utah & 5 & 81 & 16 & Yes\tabularnewline
Vermont & 22 & 1361 & 63 &\tabularnewline
Washington & 15 & 2424 & 161 &\tabularnewline
West Virginia & 2 & 128 & 72 & Yes\tabularnewline
Wisconsin & 20 & 845 & 43 &\tabularnewline
Wyoming & 11 & 238 & 21 &\tabularnewline
\bottomrule
\end{longtable}


\end{document}
